\documentclass{article}
\usepackage{graphicx}
\usepackage{listings}
\usepackage{hyperref}

\begin{document}
\title{Sneaky Snake Overview}
\author{Jeramey Tyler: tylerj2@rpi.edu}
\maketitle

\section{Required Files}

\begin{itemize}
\item{\href{https://raw.githubusercontent.com/JerameyATyler/coderDojo/master/sneak.py}{sneak.py}}
\item{\href{https://github.com/JerameyATyler/coderDojo/raw/master/__pycache__/snake.pyc}{snake.pyc}}
\item{\href{}{Optional: How to open and navigate a terminal}}
\end{itemize}

\section{Description}
The intention of this activity is to give the coders an idea of what it's like to work on a big project with lots of teams working separately. 

My intention is that this will be presented in one of two ways:
\begin{itemize}
\item{Functions can assigned to particular a particular coder or a group of coders. After finishing their assigned functions the coders can bring all of their functions together and see how they work together.}
\item{A coder or a group of coders can cherry pick functions to work on}
\end{itemize}

To keep the inner working of Snake.py a secret I've pre-compiled it to snake.pyc. This way there can be no peeking at the actual implementation. I use lambda functions to essentially toggle which function the game uses for certain tasks. This way the coders can simply comment or uncomment the appropriate line in the main function of Sneak.py. I don't intend on explaining it to the coders unless someone asks. The simplest explanation would be to simply state that those lines of code are voodoo. 

\end{document}